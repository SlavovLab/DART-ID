\def \suppmodel {{ \textbf{(a)}\; Let $\rho_{ijk}$ be the retention time for peptide $i$ in experiment $k$.  Initially the distribution of retention times across experiments may vary. To align the experiments we assume that there exists a set of \emph{canonical} retention times $\mu_i$ for all peptides and set of simple monotone increasing functions $f_k$, for each experiment $k$. $\epsilon$ is an error term expressing residual (unmodeled) variation in retention time.  As a first approximation, we assume that the observed retention times for any experiment can be well approximated using a two-segment linear regression model.\;\textbf{(b)}\; The distribution of the retention time of a PSM of peptide $i$ in experiment $k$ is modeled as a mixture model where $\lambda_{ijk}$ is the PEP for a given PSM, $\mu_{ijk}$ is the canonical retention time for a given PSM, and $\sigma_{i}$ is the standard deviation for that peptide. $\mu_{0}$ and $\sigma_{0}$ are the mean and standard deviation of the log density of all retention times, respectively.
}}

\def \model {{ \textbf{Updating error probability from spectral confidence (PEP) using retention time}\\ (\textbf{a})\; The bayesian inference framework, described more in-depth in the ... section. (\textbf{b})\; Concept for the update of the error probability, with two example observations from the same peptide and experiment. Observed RT is the observed retention time for a PSM. Predicted RT Density is the RT density of the PSM if the identification is correct, and is specified using RTs from the same peptide sequence in different experiments applied within our Alignment Model (see ... section). Null RT Density is derived from the empirical distribution of all RTs within the one experiment of the PSM. Using our framework, the PSM in Case 1 is upgraded in confidence due to its proximity to the Predicted RT Density, while the PSM in Case 2 is downgraded in confidence due to its distance from the Predicted RT Density. (\textbf{c})\; 2-D Density of PSMs, by Spectra PEP vs. Spectra + RT PEP. Dotted lines show common identification threshold of $\alpha = 0.01$. (\textbf{d})\; Fold-change increase of identifications of peptides, as a function of the identification threshold ($\alpha$ = PEP) }}